\chapter{Summary}

Haematopoiesis is the complex process of blood cell production, carried out by haematopoietic stem cells in the bone marrow. Over the human lifespan, it generates quadrillions of cells which participate in essential bodily functions such as immunity (whiteblood cells), oxygen distribution (red blood cells) and blood coagulation (platelets). However, a several conditions affect this process --- a lack of nutrients such as iron or folate can lead to alterations to the production of red and white blood cells, whereas the accumulation of somatic mutations contributes to the formation of genetically distinct subpopulations of haematopoietic stem cells (or clones) whose behaviour is partly determined by mutations conferring growth advantages (also known as driver mutations). 

When mutations in cancer-associated genes are present in the blood of healthy individuals this is known as clonal haematopoiesis, a benign condition that can progress to blood cancers such as myelodysplastic syndromes, characterized by an excess of abnormally developed (or dysplastic) cells in the bone marrow and in the blood. Blood cancers such as these are usually diagnosed by trained experts using a number of complementary analyses, including the inspection of blood cell morphology (cytomorphology) --- which can have high inter-individual variance --- and differential cell counts under the microscope.
In this work, I have studied how the haematopoietic system and blood are altered in two distinct settings --- the evolution of somatic mutations in healthy individuals and the cytomorphological alterations to blood cells in myelodysplastic syndromes.

Firstly, I studied the somatic evolution of the haematopoietic system in healthy individuals using longitudinal sequencing and single-cell-derived colonies. With simulations, I develop a computational model and apply to a cohort of elderly individuals with clonal haematopoiesis and show how i) driver genetics determines the growth rate of different clones, ii) clones appear consistently through life (with few exceptions), iii) that clones decelerate due to an increasingly competitive clonal landscape and iv) how mutations which confer greater growth advantages are associated with a greater risk of developing haematological malignancies.

Secondly, I used a cohort of digitalised whole blood smears from healthy individuals and individuals with anaemia or myelodysplastic syndromes to study how each condition leads to cytomorphological alterations through computational methods. To do this, i) I developed and implemented methods for the high-throughput detection of blood cells from digitized whole blood slides, ii) developed a cellular characterisation protocol that captures morphological features relevant for the prediction of clinically-relevant conditions and the presence of specific mutations in myelodysplastic syndromes and iii) describe novel associations between blood cell phenotype and anaemia or myelodysplastic syndrome subtypes.

By studying how haematopoietic stem cells evolve in the human body, I contributed to the understanding of early cancer development and to the broader field of human somatic evolution, and by quantitatively studying alterations to cytomorphology in clinically-relevant conditions, I showed how this can reveal novel blood cell phenotypes which can aid in diagnostic and prognostic. Both projects offer different perspectives into haematopoiesis as a dynamic process and contribute to clinical research by highlighting connections between somatic evolution in healthy individuals and cancer onset, and by discovering previously unknown cellular phenotypes-disease associations.
