\chapter{Determining the effect of repeated sampling on the theoretical limit of detection}
\label{appendix:theoretical-lod}

Across this work we sequence individuals a median of three times across their lifetime. We define a detection threshold of 0.5\% \ac{vaf} as the minimum clone size for detection on individual timepoints, but the repeated sampling leads to 0.5\% \ac{vaf} being an overestimation of the actual limit of detection (LOD) --- the size at which clones become detectable.

To show this, we simulate the repeated sampling of variants existing at a true clone proportion between 0 and 2\%. We use this proportion p as the probability parameter in a beta binomial distribution, the overdispersion  calculated using technical replicates as the overdispersion $\beta$ in the same beta binomial distribution and a coverage of 1000. Having fully parameterized this distribution ($counts \sim BB(trials=1000,\frac{p\beta}{1-p},\beta)$) we sample counts from it between 1 to 5 times. For each combination of clone size and number of samples we perform 1,000 realizations and calculate the number of detected clones at a threshold of 0.5\%. This allows us to assess the fraction of clones with a specific size which are detected if we sample them multiple times --- in other words, are able to assess the detection rate for different clone sizes and different numbers of samples.

With this, we show that, at a threshold of 0.5\% and sampling only once, we detect 14.8\% of all clones existing at 0.5\% (\supfigref{fig:theoretical-lod}). However, repeating this sampling 3 and 5 times leads to the detection of approximately 37.7\% and 54.3\% of all clones existing at 0.5\%, respectively. As such, under regular conditions --- a single sample --- we would detect 13.5\% of all clones present at 0.5\% with a detection threshold of 0.5\%. The question we should now ask is: what is the smallest possible clone size we detect at the same rate of detection --- 13.5\% --- if we increase the number of samples? Using the same set of simulations, we can calculate the likely minimal size of the detected clones, summarized in \suptableref{table:theoretical-lod}, with clones as small as 0.21\% and 0.14\% being detected with 3 and 5 samples, respectively, using the same detection rate. As such, when considering the theoretical LOD used in Figure 4k, we avoided using 0.5\% which, as we show, would be at least twice as high as the theoretical LOD obtained from simulations.

\begin{figure}[!ht]
	\placefigure{gfx/dynamics-2/theoretical_lod.pdf}
	\placecaption{Fraction of detected clones upon repeated samples/timepoints at a detection threshold of 0.5\%.}
	\label{fig:theoretical-lod}
\end{figure}

\begin{table}[!ht]
    \centering
    \caption{The minimal size of detected clones using a 0.5\% threshold and assuming that we are interested in detecting the same fraction of clones we would detect with a single sample at a detection threshold of 0.5\%.}
    \pgfplotstabletypeset[
    string type,
    columns/n/.style={
        column name=Number of samples,
        column type={C{.2\textwidth}}},
    columns/m/.style={
        column name=Minimal size of detected clones at 15.08\%,
        column type={C{.2\textwidth}}},
    every head row/.style={before row={\toprule},after row=\midrule},
    every last row/.style={after row={\toprule}},
    every odd row/.style={before row={\rowcolor[gray]{0.9}}}
    ]\theoreticalLOD
    \label{table:theoretical-lod}
\end{table}