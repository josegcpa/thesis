\chapter{Discussion}

In this dissertation, I have studied changes to haematopoiesis through the analyses of somatic evolution and cytomorphology. Regarding the former, I showed that, in \ac{ch}, different drivers are associated with distinct growth rates and age at clonal onset signatures, with clonal behaviour changing through life as the haematopoietic compartment becomes increasingly crowded. My analyses regarding the cytomorphology of blood cells showed that specific hallmarks can be determined through the use of computational methods, a process that has so far been driven by experts in haematology and cellular morphology, by developing a not only a framework for the detection and characterization of individual blood cells, but also a \ac{mil}-based methodology to cluster \ac{wbc} and \ac{rbc} into clinically-relevant cell types. 
\paragraph{\Ac{ch} and its role in early detection.} It is a well known fact that \ac{ch} is associated with progression to haematological malignancies and death \cite{Genovese2014-eu,Jaiswal2014-rl,Abelson2018-wh}. However, translating this knowledge to the clinic is complicated --- most cases of \ac{ch} will not be associated with transformation to a myeloid malignancy, making it necessary to better understand this process. By showing here how specific drivers in \ac{ch} are not only associated with relatively well-conserved dynamics, I also demonstrate that mutations associated with an increased risk of progression grow at a much faster rate in healthy individuals. 

This knowledge can potentially help clinicians by revealing which mutations should be of concern and whether they should be tracked through time. However, it still holds that this knowledge is vastly incomplete --- several studies, including the work here presented, have highlighted how little is actually known about \ac{ch} \cite{Poon2020-ek,Mitchell2021-zl,Fabre2021-uw}. Future studies into \ac{ch} need to focus much more on the discovery of novel variants, creating a more complete picture of \ac{ch} before the effective and economically-feasible tracking of most \ac{ch} variants becomes a reality; for this, approaches such as those applied by \etal{Genovese} and \etal{Jaiswal}, where blood whole-exome sequences were obtained for very large cohorts and used to identify blood clones in healthy individuals \cite{Jaiswal2014-rl,Genovese2014-eu}, can be adapted by increasing the sequencing depth to further reveal recurring drivers of \ac{ch}. Additionally, studies which use single-cell colonies to explore \ac{ch}, such as the work here presented and that by \etal{Mitchell} can be extremely useful --- not only do they allow the practical exploration of the lifelong trajectories of different expansions, they can allow, given sufficient magnitude, for the tentative identification of novel drivers of \ac{ch}. However, the cost of deriving a sufficient number of single-cell colonies can act as a considerable obstacle. \etal{Poon} overcame the problem of data availability by using large cohorts where synonymous mutations could be used to track clonal expansions --- by doing so, and under the fair assumption that synonymous mutations can only expand by being passengers in clonal expansions, they were able to identify novel putative drivers. This knowledge can be used to create large panels of potential drivers which can then be further validated. With this knowledge, single-cell colonies could be used to establish the lifelong dynamics of putative drivers. Nonetheless, it is necessary to consider the recent evidence showing the high prevalence of \ac{mca} and copy number alterations in \ac{ch} --- these can act as powerful confounders, especially considering that some mutations are associated with copy number alterations that directly affect the gene \cite{Gao2021-ph,Saiki2021-sq}. Finally, it is worth noting that the collection of clinical data --- such as blood counts and behavioural patterns --- can help illuminate the mechanisms underlying specific expansions or even stratify individuals regarding the \ac{ch} risk and risk of progression \cite{Dawoud2020-af,Abelson2018-wh}.
\paragraph{Computational cytomorphology --- a potential path to the clinic.} In this dissertation I show how different clinically-relevant conditions can be predicted from \ac{wbs} using only computational methods. Additionally, using a \ac{mil} protocol developed by me, I show how new cellular archetypes may be discovered in situations where little annotation is required, effectively decreasing the necessary input from haematological experts. However, generalization to another cohort, digitalized with a different scanner and prepared under different standards, is still unsatisfactory --- the most logical path towards a possible solution for this lack of generalization is the use of multi-centre training cohorts. Additionally, regarding the characterization of blood cells, improvements can be made --- large databases of annotated cells can enable the extraction of morphologically-meaningful features with no need for feature design by training classification \ac{cnn} models and using the feature vectors produced by these models as morphological descriptors of different cells.

\Ac{ml} and \ac{dl} systems enable the high-throughput and reproducible analysis of \ac{wbs} in a clinical context. However, particular hurdles are still in the way --- the absence of very large, multi-centre cohorts impedes the realistic assessment of predictive power. Additionally, most studies are retrospective (using available and previously stored data), which is not representative of point-of-care where predictions are expected on a regular basis for new slides \cite{Eckardt2020-fp}. Large multi-centre cohorts also enable a more realistic learning of the true disease signal, which should be independent of cohort-specific noise. An additional factor to consider is that diagnostic conventions change --- how these prediction algorithms change is also worthy of consideration: if an augmented intelligence setting is considered, where the predictive algorithm is merely an assistant to the expert which is expected to confirm the diagnoses, a reinforcement learning strategy should be considered, where the algorithm is iteratively refined by a team of experts. Federated learning, a decentralized \ac{ml} paradigm that permits guarantees data-privacy and better data governance \cite{Rieke2020-hl}, could be combined with large multi-centre databases and reinforcement learning to create a powerful semi-automated diagnostic system capable of constant updates. However, if diagnostic criteria change, new training cohorts are possibly necessary --- in the worst case scenario, a disease subtype may be stratified further, requiring expert annotation or confirmation (it may also be possible that a classification algorithm will already, implicitly, cluster disease subtypes within its feature space). This last circumstance would require the retraining of these models.

Finally, it is worth considering whether generalization to other digitalisers should be an objective of the field. Companies, such as Siemens Healthineers, working on the automated analysis of \ac{wbs} focus, for now, exclusively on the differential counting of \ac{wbc} and generic cellular detection of \ac{wbs} with some abnormality detection capabilities \cite{cellavision,advia-120}. These systems, by also incorporating the automatic preparation of a \ac{wbs} from a blood sample \cite{advia-120}, eliminate most of the variability associated with slide preparation and digitalization. In realistic terms, this simple solution --- training models which are optimized for specific systems of slide preparation and digitalization and testing them prospectively --- will provide the strongest evidence for their generalization as a clinically applicable and transferrable solution. Models for disease prediction from \ac{wbs} --- such as those developed in this dissertation --- become much more applicable in these circumstances. Nonetheless, it should be noted that specific sources of noise, associated with the specific composition of different populations, may still be a problem for these models --- it has been shown that incidence and 5-year survival of blood cancers, or the incidence of specific haemoglobinopathies (such as thalassemia, a condition characterized by decreased or defective haemoglobin production), differs by ethnicity and race \cite{Kirtane2017-dh,Lorey1996-yc}. The aetiology of such differences is unknown, with factors such as diet, socio-economic conditions and biology being possible causes, and the changes they elicit on the cytomorphology of blood are far from being studied. Approaches such as the one presented here, where morphometric trends and cell types are established in association to specific conditions, could help reveal the morphological signatures of different ethnicities and races, thus creating better informed diagnostic practices.